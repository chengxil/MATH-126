%!TEX root = ./main.tex
\section*{Logistics}
\begin{question}
	Why study PDE?
\end{question}
\begin{answer}
	TL;DR. It's useful.
\end{answer}
\begin{note}
	Office hour : MWF 9-10AM 895 Evans, GSI Office hour : MW 1-3 PM 1049 Evans
\end{note}
\section{where PDEs Come From}
\subsection{What is a partial differential equation?}
\begin{example}
	This is an example of ODE:
	\[ u  = u(x), \qquad \frac{d}{dx} u = u.\]
\end{example}
\begin{example}
	A PDE consist of the form
	\[ u = u(x_1, x_2, \ldots, x_d), \qquad u_{x_k} = \frac{\partial u}{\partial x_k}\]
	Where $x_i$ are scalars.
\end{example}
\begin{example}
	The most general form of a PDE of first order in two dimension, say $u = u(x,y)$ and of the form
	\[ F(x,y,u(x,y), u_x(x,y),u_y(x,y)) = 0 , \quad \mathrm{ or } \quad F(x,y,u, u_x, u_y) = 0\]

\end{example}
\begin{example}
	The most general form of a PDE of second order in two dimension, say $u = u(x,y)$ and of the form
	\[ G(x,y,u, u_x, u_y, u_{xx}, u_{yy}, u_{xy}) = 0 \]
\end{example}
\begin{definition}
	A vector $x$ is defined as
	\[ x = \vec x = (x_1, x_2, \ldots, x_n).\]
\end{definition}
\begin{definition}
	Let $u$ be a function of vector $x$ of $n$-dimension. The gradient of $u$ is denoted as
	\[ \nabla u = (u_{x_1}, u_{x_2}, \ldots, u_{x_n})\]
\end{definition}
\begin{example}
	\begin{enumerate}
		\item Linear transport equation $ \quad u_t + bu_t = 0, \quad b \in \b R$

		\item Burgher's Equation 
		$\quad u_t + u \cdot u_x = 0$
		\item Laplace's Equation
		$\quad u_{xx} + u_{yy} = 0$
		\item Hermite Equation
		$\quad -(u_{xx} + u_{yy}) = \lambda u, \quad \lambda \in \b R$
		\item Wave with interaction
		$\quad u_{tt} - u_{xx} + u^3 = 0$
		\item Linear diffusion with source
		$\quad u_t - u_{xx} - f(x,t) = 0$
		\item Schroedinger's equation 
		$\quad u_t - i \cdot u_{xx} = 0$
	\end{enumerate}
\end{example}
\begin{example}[Cauchy-Riemann Equation]
	\[ \left\{ \begin{array}{cc}
		u_x &= u_y \\
		u_y &= -u_x
	\end{array} \right.\]
\end{example}
\begin{definition}[Digression to Linear Algebra]
	Let $\mathscr L$ be a operator in a function space $V$. $\s L$ is linear if  
	\[\mathscr L(u + v) = \mathscr L (u) + \mathscr L(v), \quad \mathscr L(cu) = c\mathscr L(u) \qquad  \forall v,u \in V, \quad \forall c \in \b F.\]
\end{definition}
\begin{definition}
	A PDE is called homogeneous linear PDE if it's of the form $\mathscr L(u) = 0$. If it's the form $\mathscr L(u) = f$, then it's called inhomogeneous PDE.
\end{definition} 
\begin{remark}
	Things that we are interested in
	\begin{enumerate}
		\item Find analytical formulas for some specific PDE's
		\item Well-possessedness
		\begin{itemize}
			\item Existence (Does there exists a solution?)
			\item Uniqueness (Is this the only solution?)
			\item Stability (If I change the data slightly, does the solution changes just by a little bit?)
		\end{itemize}
		\item Predicting qualitative (and sometimes quantitative) behavior of the solution without having a solution formula. 
		\item Devise an analyze numerical algorithms to approximate solutions.
	\end{enumerate}
\end{remark}
\begin{example}
	Consider the equation
	\[ \cos(xy) u_x + \sin \left( e^x \right) u_yy = e^{x^2\sin (y)} \]
	Let 
	\[ \mathscr L(u(x,y)) = \cos(xy) u_x + \sin \left( e^x \right) u_{yy} \]
	$\mathscr L$ is a linear operator, so the PDE is an inhomogeneous linear PDE.
\end{example}
\begin{theorem}[Principle of superposition]
	Let $u_1, u_2, \ldots, u_n$ be solutions of $\mathscr L(u_k) = 0$, and let $c_1, c_2, \ldots, c_n$ be scalars. then
	\[ u(x) = \sum_{i = 1}^{n} c_i u_i(x) \quad \mathrm{ solves } \quad \mathscr L(u) = 0\]
\end{theorem}
\begin{example}[Cool problem]
	\[ \left\{\begin{array}{rl}
		u_t + u \cdot \nabla u - \mu \triangle u &= - \nabla p \\
		\mathrm{div}\, u	&= 0 \end{array} \right.\]
		where $u = u(x,y,z,t)$, $u = \begin{pmatrix}
			u_1 \\
			u_2 \\
			u_3 \\
		\end{pmatrix}$ velocity field, where $p$ is pressure and $\mu$ is the viscosity of the liquid.
\end{example}