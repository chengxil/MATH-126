%!TEX root = ./main.tex
\section{Waves and Diffusion}
\subsection{The wave equation}
Recall that the wave equation in in dimension
\[ u_{tt} = c^2 u_{xx} \quad \mathrm{for} \quad -\infty < x,t < \infty\]
Observe that
\[ 0 = u_{tt} - c^2 u_{xx} = \left( \frac{\partial }{\partial t} - c \frac{\partial }{\partial x} \right) \left( \frac{\partial }{\partial t} + c \frac{\partial }{\partial x} \right)v  =: v \]
We can equivalently write our second order PDE as 
\[\left\{ \begin{array}{rl} 
	u_{t} + c u_{x} &= v \quad (2) \\
	v_t - c v_x &= 0 \quad (1) \\
\end{array} \right. \]
We know that the general solution to (1) is
\[ v(x,t) = h(x + ct)\]
where $h$ is an arbitrary function of one variable. Then substituting $v$ into (2) gives us
\[ u_t + cu_x = h(x + ct)\]
We know that one particular solution is given by $u(x,t) = f(x,t)$, where 
\[ f'(s) = \frac{h(s)}{2t}\]
To that, we can add any of the homogeneous solution
\[ u_t + c u_x = 0 \implies u(x,t) = f(x + ct) + g(x - ct)\]
Hence we have shown that 
\[ u(x,t) = f(x + ct)  - g(x - ct)\]
where $f,g$ are arbitrary function.
\subsubsection{Characteristic coordinates}
Take
\[ \xi = x + ct \qquad \eta = x - ct\]
By the chain rule we have
\[ \partial_x = \frac{\partial}{\partial x} = \frac{\partial }{\partial \xi} \frac{\partial \xi}{\partial x} + \frac{\partial }{\partial \eta} \frac{\partial \eta}{\partial x} = \partial_\xi + \partial_\eta\]
\[ \partial_t = \frac{\partial}{\partial t} = \frac{\partial }{\partial \xi} \frac{\partial \xi}{\partial t} + \frac{\partial }{\partial \eta} \frac{\partial \eta}{\partial t} = c\partial_\xi - c\partial_\eta \]
Hence
\[ \partial_t - c \partial_x = -2c \partial_\eta \qquad \partial_t + c \partial_x = 2c \partial_\xi\]
SO the wave equation is of the form
\[ 0 = (\partial _t - c\partial_x)(\partial_t + cd_x)u = (-4c \partial_\xi)(2c\partial_\eta)u = -4c^2\partial u_{\xi \eta}\]
Since $-4c^2 \neq 0$, we have $u_{\xi \eta} = 0$.
So $u(x,y) = f(\xi) + g(\eta)$.
\subsubsection{Initial Value Problem}
Take 
\[ \left\{ \begin{array}{rl} 
	u_{tt}  &= c^2 u_{xx}   \\
	u(x,0) &= \phi(x) \\
	u_t(x,0) &= \psi(x)
\end{array} \right. \]
where $\phi(x) = \sin x$, and $\psi(x) = 0$.
From the general solution we put $t = 0$ and obtain.
\[ \phi(x) = f(x) + g(x)\]
differential by $t$ we get
\[ \psi(x) = cf'(x) - cg'(x)\]
differentiate $\phi$ and divide $\psi$ by $c$ we get
\[ \phi's = f' + g' \qquad \frac1c \psi = f' - g'\]
Solving for $f'$ and $g'$ gives us
\[ f' = \frac 12 \left( \phi' + \frac \psi c \right) \qquad g' = \frac 12 \left( \phi' - \frac \psi c \right)  \]
Integrate with respect to $s$ gives us
\[ f(s) = \frac 12 \phi(s) + \frac 1{2c} \int_0^s \psi + A \qquad f(s) = \frac 12 \phi(s) - \frac 1{2c} \int_0^s \psi + B\]
where $A,B$ are constants. Since $\phi(x) = f(x) + g(x)$, we have $A + B = 0$. Let $s = x + ct$ and $s = x - ct$ we get
\[u(x, t)=\frac{1}{2} \phi(x+c t)+\frac{1}{2 c} \int_{0}^{x+c t} \psi+\frac{1}{2} \phi(x-c t)-\frac{1}{2 c} \int_{0}^{x-c t} \psi\]
which is reduced to 
\[ \boxed{u(x,t) = \frac12 \left[ \phi(x + ct) + \phi(x - ct) \right] + \frac 1 {2c} \int_{x - ct}^{x + ct} \psi(s) d(s)}\]
\begin{example}
	Take $\phi = 0$ and $\psi = \cos x$. Solve for the wave equation.
\end{example}

\subsection{Causality and Energy}


